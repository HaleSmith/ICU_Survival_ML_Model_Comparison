\documentclass[11pt]{article}
\usepackage{framed}

\usepackage{fullpage}
\usepackage{amsmath}
\usepackage{listings}
\usepackage{tikz}
\usepackage{amsmath}
\lstset{language=C}
\begin{document}

\title{Survivability of ICU Patients with Severe Sepsis/Septic Shock}
\author{Jeremy B. Crowe - crowe.jb@gmail.com}
\maketitle

\section{Domain Background}
\subsection{Overview}
Sepsis is a serious, life threatening condition that is caused by an overwhelming immune response to an infection. Infection is often bacterial however it is possible for sepsis to be associated with: fungal, viral, parasitic, as well as bacterial infections.

Sepsis is due to the body's response, not only the effects of the infection itself. White blood cells release an array of chemicals to fight the infection which trigger systemic inflammation, vasodilation, permeability of vessels, and intracellular fluid build up. These ``leaky vessels'' deplete the body of coagulation factors. The increased fluid build up and decreased blood pressure result in a lack of oxygenation of tissue, known as shock.

If sepsis is not treated quickly or with enough direct care, multiple organ dysfunction can occur which can result in kidney failure, liver failure, heart failure, acute respiratory distress, etc. The speed at which treatment is administered and types of treatments are directly correlated with survivability of this severe condition. The speed of treatment has been found to be more important than the age of the patient~\cite{survival2}. Each hour of delay in antimicrobial administration over the initial 6 hours is associated with an average decrease in survival rate of 7.6\%~\cite{survival}.

Certain treatments may be highly effective, therefore it is important to pay special attention to several things: the type of infection and type of antibiotics used, the blood pressure and whether or not vasopressors were used, correct amount of IV fluid provided to patient. The correct type of antibiotic and appropriate administering of vasopressors can reduce the chance of organ failure and mortality greatly~\cite{pressors}.

\subsection{Why This is Important}

Sepsis is partially responsible for 6\% of all deaths in the United States~\cite{cdc}. Sepsis is a highly variable condition caused by numerous infections and pre-existing conditions. A patient may arrive at the ICU with sepsis or may be diagnosed during the stay. There is no diagnostic test to confirm or deny the existence of such a condition in a patient. It is diagnosed based on the clinical judgment of physician. Regardless of the cause, the speed at which treatment is administered and the types of treatments are critical in stabilizing a patients blood pressure, body temperature, eliminate infection, and keep cardiac output stable. Due to the variable, and difficult nature of the condition a machine learning approach may be able to aid in predicting the severity of the condition and the effectiveness of potential treatments.


\section{Problem}
Sepsis is a complex and indirect condition that is very dangerous and difficult to diagnose. Providing an anti-microbial/anti-biotic treatment early can be critical; often non-optimal antibiotics are prescribed prior to the lab culture results are returned. Finding the most prevalent cause for infection and most effective anti-microbial treatment can save time and lives in the initial treatment of a patient suffering from sepsis.

It is useful to know the severity of the condition upon arrival to better treat patients. If a machine learning algorithm can be used to assess a patient at admittance and predict the probability of survival, it would help hospital staff in prioritizing patients.

\section{Datasets and Inputs}
For this study I will be using the MIMIC-III dataset(https://mimic.physionet.org/). This dataset is freely accessible and suggested for use by Udacity. I will be utilizing the data on more than 40,000 individual patients. 1184 patients were admitted with sepsis and thousands more diagnosed by the end of their stay. I will be focusing on the patients admitted with sepsis since I have a time of admittance that will aid in finding more effective treatments. Diagnosis data is tied to icd-9 billing codes that do not have a time associated with them unfortunately.

It was required to take a short course to gain access to this data by demonstrating some understanding of law and ethics in dealing with this de-identified dataset.

I have downloaded the data locally as csv files. I then modified a script provided by MIT to load the entries into an Amazon RDS Postgres database. This database and server is secured and is HIPAA compliant. I then have spun up an Amazon compute EC2 instance to do most of the heavy lifting. I will be using many tables from the database but the most critical are the following:
\begin{enumerate}
\item admissions: the admissions table provides basic patient information on admission including the initial diagnosis, sex, insurance type, etc.
\item diagnoses\_icd: this table provides all diagnoses assigned to a patient during a hospital stay. Unfortunately this is assigned at the end and cannot be given a time of diagnosis. A sequence is provided however.
\item microbiologyevents: this table provides information about whether or not an infection is present, how it was obtained, and when.
\item labevents: this provides measurements such as blood pressure, and information from blood tests.
\item prescriptions: drugs given to patient along with the time and amount.
\item patients: this provides basic information about the patient including date of birth, date of death (if applicable), gender, etc.
\end{enumerate}


\section{Solution}
I am writing a machine learning algorithm to take basic information about the patient including the type of bacterial infection, type of antibiotic treatment, time between admission and first dose, and survival of patient. The goal is to predict the survival of a patient based off of this information shortly after admittance. As the project progresses it may require me to include more complex information such as blood pressure measurements and categorical data about whether vasopressors were used to combat low bp. 
The end goal is to be able to predict the survivability of a patient but as a corollary make predictions about the best tailored treatments for each patient. 

\section{Benchmark and Evaluation}
There are few studies into the predictability of patient survival from severe sepsis. There are also many factors that are important in how the prediction is made. I will be focusing on the following research: ``Predicting survival of patients with sepsis by use of regression and neural network models'' by J.R. Flanagan, et al. \begin{quotation}
  ``The development of multiple organ failure after the diagnosis of sepsis was predicted accurately (81.5\%) with either the MLR or the NN model. Both the MLR and the NN methods depended on the interpretation of a likelihood quantity, requiring the choice of a threshold to make a survival prediction.''\cite{sepsisresearch}
\end{quotation}

While this study was done in 1996, there have been few followup studies. The few that I found had a similar success rate. Due to the detail of this study I am focusing on the work of Flanagan et al. for my benchmark. I have requested full access to their research via ResearchGateway.

\section{Design}
Due to the nature of the problem and the relatively large number of features that will likely be involved, I will be comparing the results of a Neural Network, Adaboost with a decision stump as its base estimator or extreme gradient boosting. Only Adaboost is possible in SKlearn, so I may need special permission to use Tensorflow for a multi-layer Neural Network and XGBoost for extreme gradient boosting.

The data is spread across multiple database tables. For most algorithms I will need to join this data into a large single entry if not using time series data. For reusability I will likely create a new table with this joined information for easy querying.


\bibliography{citations}{}
\bibliographystyle{ieeetr}

\end{document}